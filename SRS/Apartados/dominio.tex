\section{Información del dominio del problema}

%\section{Problem domain information}
  l
\begin{textoazul}
 
 Esta sección obligatoria debe contener información relativa al dominio del problema que permita comprender los \textbf{conceptos básicos} del mismo al lector del documento. Se divide en las secciones que se describen a continuación.
 
 %this mandatory section shoudl contains problem domain information 
\end{textoazul}
 
\subsection{Introducción al Dominio del Problema}

 \begin{textoazul}
 Se trata de dar una visión general del conjunto de conceptos que se manejan en la organización para la que se va a desarrollar el sistema software. Pueden incluirse diagramas u otro elemento multimedia si se considera oportuno para facilitar su comprensión.
 
 % Overview of the concepts managed by the organization. You can include diagrams or any other description method
 \end{textoazul}
 
\subsection{Glosario de Términos}
%\subsection{Glossary}

 \begin{textoazul}
 Esta sección debe contener una lista ordenada alfabéticamente de los principales términos, acrónimos y abreviaturas específicos del dominio del problema, especialmente de los que se considere que su significado deba ser aclarado. Cada término, acrónimo o abreviatura deberá acompañarse de su definición y se podrá adjuntar material multimedia que facilite su comprensión como fotografías, documentos escaneados, diagramas o incluso vídeo o sonido en el caso de que el formato de la ERS lo permita

% This section should contain an alphabetical list of key terms, acronyms and abbreviations specific problem domain, especially those deemed that its meaning should be clarified. Each term, acronym or abbreviation must be accompanied by its definition and multimedia material may be attached to facilitate its understanding as photographs, scanned documents, diagrams or even video or sound in case the SRS format permits
\end{textoazul}
